\documentclass{article}
\usepackage[utf8]{inputenc}

\title{Tugas Rangkuman Database 1 }
\author{Krisna Bayu Pratama \\ NPM : 1194020 \\ Prodi/kelas : D4 TI/1A \\ Dosen Pengampu : Syafrial Fachri Pane, S.T, M.T.I, EBDP }
\date{KAMIS, 27 February 2020}

\usepackage{natbib}
\usepackage{graphicx}

\begin{document}

\maketitle

\section{Tentang Database}
\paragraph{}
Database merupakan pengumpulan data secara sistematik yang merupakan reprentasi dari dunia nyata,yang datanya harus valid dan terdapat sumber data yang diterima, dan sebisa mungkin agar tidak ada data yang redudancy (muncul data yang sama), sehingga ketika kita menginginkan untuk mendapatkan informasi tentang data kita tidak kebingungan ketika ada data yang ganda.
\paragraph{}
Dalam Database terdapat berbagai software yang dapat digunakan antara lain : \\
1. Ms. Access \\
2. Ms. Sql\\
3. Oracle\\
4. MySql\\
5. Postgre Sql\\
6. Firebird\\
Dalam aplikasi diatas terdapat berbagai keunggulan dan kekurangan dalam pembuatan Database, software yang sering digunakan antara lain MySql dan Oracle.

\section{Apa Yang Membuat Database harus digunakan?}
\paragraph{}
a. Database merupakan salah satu cara yang efisien dalam mengumpulkan dan mengelompokkan data sebagai pusat informasi.\\
b. Database lebih efisien dalam pengelompokan data.\\
c. Sebagai untuk mencegah redududancy.\\
d. Berbeda dengan database tradisional, karena ketika terjadi suatu hal seperti kebakaran, atau kita memerlukan data dengan cepat, kita tidak kehilangan waktu sia sia karena lebih cepat dapat informasi yang harus segera didapat.

\end{document}