\documentclass[12pt]{article}
\author{oleh Deriska Fadilla Musdalifa\\1194010\\D4 Teknik Informatika\\1A}
\date{Thursday 27/02/2020}
\title{TUGAS MATA KULIAH BASIS DATA 1}
\begin{document}
\maketitle
Basis data atau biasa disebut dengan database adalah suatu kumpulan data/ informasi untuk menyimpan data terstruktur dan tidak struktur yang dikelompokkan dalam suatu tempat berdasarkan fakta agar tidak terjadinya redudance (pengulangan) yang tersimpan dalam komputer serta data tersebut harus saling berelasi sehingga dapat diperiksa menggunakan suatu program komputer untuk memperoleh informasi dari database tersebut. 

DBMS (Database Management Sistem) yaitu perangkat lunak/ software yang digunakan untuk membangun basis data contoh dari DBMS yaitu MySQL, Oracle, dll. Tempat penyimpanan data berada di MySQL. Dalam flowchat database mempunyai symbol berupa tabung, di dalam tabung tersebut terdapat beberapa tabel/ entitas yang diperlukan dalam database, lalu dikirimkan ke ISP/Cloud untuk menyimpan database. 

Fungsi normalisasi adalah untuk memisahkan antar tabel/ entitas sesuai dengan arti khas tabel itu sendiri. Normalisasi membutuhkan relasi, sesuai dengan Bahasa inggrisnya relasi adalah hubungan, artinya relasi adalah jalur penghubung antar tabel satu dengan tabel lainnya. 
Fungsi dan tujuan dari basis data : 

\begin{itemize}
  \item Untuk menghindari redudance/data ganda.
  \item Penyimpanan data.
  \item Mengelompokkan data untuk mempermudah identifikasi data.
  \item Mempermudah dalam akses data.
\end{itemize}

Perbedaan basis data tradisional dan modern, pada database modern mudah diakses, tempatnya luas, mudah untuk pengembangan, sedangkan pada basis data tradisional susah diakses, sempit, susah untuk pengembangan.

Contoh basis data dalam kehidupan sehari hari adalah meja belajar, di dalam meja belajar terdapat banyak slot untuk menyimpan sebuah barang, bagian slot yang besar merupakan wadah untuk barang yang besar misalnya tas, lalu pada slot yang kecil untuk menyimpan barang ang kecil, lalu terdapat rak besar dan kecil, rak besar bisa untuk buku pelajaran sementara rak kecil bisa untuk buku bacaan ringan seperti novel, buku cerita, dll. 

\end{document}
