\documentclass{article}
\usepackage[utf8]{inputenc}

\title{Basis data 1}
\author{azizyoy22 }
\date{February 2020}

\begin{document}

\maketitle

\section{Pengertian Basis Data}
Basis data  atau Database ialah kumpulan data berupa angka, symbol atau dari beberapa data  yang dikumpulkan dari beberapa informasi yang di bentuk  didalam satu tempat dimana datanya saling tersinkronisasi yang di kontrol oleh satu tabel atau satu data dimana dari semua susunan tersebut akan terbentuk suatu struktur sitem yang baik dalam membantu pekerjaan kita.

\section{Tujuan Basis Data}

Tujuan penyimpanan dari  Basis Data  atau Database supaya data kita terumpulkan secara sistematis dan terhindar dari data ganda (regulations).Berikut tempat penyimpanan Date Base :
\begin{itemize}
  \item Mysql.
  \item oracle.
  \item phpMyAdmin dan lain lain.
\end{itemize}
selain untuk menyimpan secara sistematis dan terhindar dari data ganda ada beberapa tujuan dari penggunaan Basis data atau Database :
\begin{itemize}
    \item Mempermudahkan identitas data
    \item Mempermudah dalam mengakses, mengedit, dan menghapus data
    \item Menjaga kualitas data dan informasi agar tetap sama dan akurat
    \item Mendukung aplikasi yang membutuhkan ruang penyimpanan yang cukup banyak
\end{itemize}
\end{document}
