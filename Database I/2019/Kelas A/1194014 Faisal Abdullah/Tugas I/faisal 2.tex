\documentclass[10pt]{article}
\usepackage[latin1]{inputenc}

\title{Rangkuman Materi Basis Data I}
\author{Faisal Abdullah }
\date{27 Februari 2020  }

\usepackage{natbib}
\usepackage{graphicx}

\begin{document}

\maketitle
\begin{center}
    \includegraphics[]{download.png}
\end{center}

\newpage

\section*{Basis Data}
\paragraph{}Basis data adalah kumpulan data yang bisa berupa angka,huruf,gambar yang saling berhubungan dan faktual kemudian disimpan dalam suatu tempat agar dapat diolah menjadi informasi
\paragraph{}Basis data dibagi menjadi dua yaitu modern dan tradsional.

\section*{Kelebihan Database modern}
\begin{itemize}
  \item Menghindari data agar tidak redudansi/ data ganda
  \item Menghindari data agar tidak memakan lebih banyak ruang
  \item Menghindari data agar tidak rusak karena kesalahan manusia
  \item Agar mudah dicari
  \item Memudahkan data agar mudah di back up
 \end{itemize}
 
  \section*{Kekurangan Database mordern}
\begin{itemize}
  \item Data bisa diretas oleh seseorang
  \item Memerlukan tenaga ahli
 \end{itemize}
 
 \section*{Kelebihan Database modern}
\begin{itemize}
  \item Menghindari data agar tidak redudansi/ data ganda
  \item Menghindari data agar tidak memakan lebih banyak ruang
  \item Menghindari data agar tidak rusak karena kesalahan manusia
  \item Agar mudah dicari
  \item Memudahkan data agar mudah di back up
 \end{itemize}

\newpage
 
 \section*{Kekurangan Database tradisional}
\begin{itemize}
  \item Data lebih memakan banyak ruang
  \item Data bisa lebih mudah rusak karena kesalahan manusia
  \item Bisa terjadi redudansi
  \item Sulit dicari
  \item Data sulit di Back up
 \end{itemize}
 
 \section*{Software Untuk Basis Data}
 \begin{itemize}
  \item MySQL
  \item Oracle
  \item Microsoft Acces
  \item Firebird
 \end{itemize}
 

\section*{Tahapan Normalisasi}
\paragraph{Tahapan normalisasi dilakukan agar data tidak redudansi/data nya ganda}

\section*{Table}
\paragraph{Table adalah kumpulan data yang ada di dalam baris dan kolom}

\section*{Contoh penerapan Database tradisional}
 \begin{itemize}
  \item Lemari
  \item Kulkas
  \item Dompet
 \end{itemize}
 
 \newpage
 \section*{Contoh penerapan Database modern}
 \begin{itemize}
  \item Kepegawaian
  \item Reservasi
  \item Layanan pelanggan
 \end{itemize}


\end{document}
