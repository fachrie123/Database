\documentclass[12pt]{article}
\usepackage{amsmath}
\usepackage{graphicx}
\usepackage{hyperref}
\usepackage[latin1]{inputenc}

\title{Rangkuman Materi Data Base}
\author{Zhunnun An Nafri}
\date{27/02/20}

\begin{document}
\maketitle

Data Base adalah kumpulan dari beberapa data yang di dapatkan dari beberapa informasi yang di buat dalam bentuk tabel atau di sebut dengan pengelompokan yang nanti tabelnya akan saling berelasi (berhubungan) yang akan di control oleh satu tabel untuk mengkontrol tabel-tabel yang lain.Setelah di bentuknya Data Base maka akn terbentuklah sistem baik itu dalam pembuatan website,aplikasi,game dan lain-lain. 

Tujuan penyimpanan dari Data Base agar data yang kita kumpulkan sistematis agar terhindar dari data yang sama (regulations).Berikut tempat penyimpanan Date Base ;

\begin{itemize}
  \item Mysql.
  \item oracle.
  \item phpMyAdmin dan lain lain.
  
  
\end{itemize}
\end{document}


