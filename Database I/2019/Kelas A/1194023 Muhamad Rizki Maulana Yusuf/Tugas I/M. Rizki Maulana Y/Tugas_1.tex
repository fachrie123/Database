\documentclass[a4paper,12pt]{article}
\usepackage{graphicx}
\title{TUGAS 1 DATABASE}
\author{M.Rizki Maulana Y. \\\\ NPM:1194023 \\\\ Kelas:D4 TI 1A \\\\ Dosen Pengampu:Syafirial Fachri Pane,S.T.,M.T.I.,EBDP}
\begin{document}
\maketitle
\begin{center}
\begin{tabular}{c c}
\end{tabular}
\end{center}
\section{PENGERTIAN DATABASE}
\paragraph{}
Database atau juga basis data adalah kumpulan informasi yang berbentuk angka,huruf,gambar,dan bilangan dan di simpan dalam suatu komputer.
\paragraph{}
DataBase awalnya berasal dari himpunan matematika.        Informasi di  olah menjadi data yang valid dengan mengecek kepastian informasi yang dikumpulkan.
\section{TUJUAN DATABASE}
\paragraph{}
Tujuan Database untuk menghindari adanya data yang sama atau ganda (REDUDANSI) dan agar data juga sistematis dan terstruktur.
\section{JENIS DATABASE}
\begin{itemize}
\item MySQL
\item Oracel
\item MS. Access
\item dBase
\item Visual Foxpro
\item Chipper
\item Firebird
\item Postgre SQL
\item IBM DB2
\item Microsoft SQL server
\end{itemize}
\section{PERBEDAAN DATABASE DULU DAN SEKARANG}
\begin{itemize}
\item Dulu
\paragraph{}
Dulu database tidak efisien sekarang dan juga banyak memakan tempat dan keamanannya kurang karna database masih berbentuk arsip dan juga data rawan hilang juga rusak.
\begin{itemize}
\item Sekarang
\end{itemize}
\paragraph{}
Sekarang Database sangat efisien, sistematis dan terstruktur yang di simpan dengan aman dan juga sangat fleksibel.
\end{itemize}
\end{document}
